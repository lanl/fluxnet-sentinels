\documentclass{article}

\usepackage{multirow}
\usepackage{array}
\usepackage{longtable}
\usepackage{booktabs}
\usepackage{graphicx}
\usepackage{inputenc}
\usepackage{geometry}
\usepackage{subfig}
\usepackage{float}

\usepackage{ragged2e}  % for '\RaggedRight' macro
\usepackage{calc}      % for '\widthof' macro
\newcolumntype{P}[1]{>{\RaggedRight\hspace{0pt}}p{#1}}
\newcolumntype{L}{>{\RaggedRight}X}

\begin{document}

\noindent Table S0: Variable definitions
\pagenumbering{gobble}

\begin{table}[H]
    \centering
    \begin{tabular}{@{}lll@{}}
    \toprule
    Variable & Definition & Type \\ \midrule
    co2 & Carbon Dioxide & Dependent \\
    le & Latent Heat Flux & Dependent \\
    h & Sensible Heat Flux & Dependent \\
    fc & Carbon Dioxide Flux & Dependent \\ \midrule
    ta & Air Temperature & Independent \\
    pa & Atmospheric Temperature & Independent \\
    ws & Wind Speed & Independent \\
    ppfd & Photon Flux Density & Independent \\
    p & Precipitation & Independent \\
    netrad & Net Solar Radiation & Independent \\
    rh & Relative Humidity & Independent \\ \bottomrule
    \end{tabular}
    \end{table}

\noindent Table S1: All possible pairwise combinations of dependent and independent variables for the IRE case study (n=51). $R^2$ is the coefficient of determination of a linear model fit to the two covariates for the period of the abnormal release event (corresponding to 'during' in Fig. 2). Effect is the event effect size of the interaction term 'during' the known release period matching the quantile of the distribution as specified by event quantile effect hyperparameter.

\pagenumbering{gobble}

\begin{longtable}[]{@{}
  >{\raggedright\arraybackslash}p{(\linewidth - 8\tabcolsep) * \real{0.1667}}
  >{\raggedright\arraybackslash}p{(\linewidth - 8\tabcolsep) * \real{0.2083}}
  >{\raggedright\arraybackslash}p{(\linewidth - 8\tabcolsep) * \real{0.1458}}
  >{\raggedright\arraybackslash}p{(\linewidth - 8\tabcolsep) * \real{0.2292}}
  >{\raggedright\arraybackslash}p{(\linewidth - 8\tabcolsep) * \real{0.1875}}@{}}
\toprule\noalign{}
\begin{minipage}[b]{\linewidth}\raggedright
Dep
\end{minipage}
&
\begin{minipage}[b]{\linewidth}\raggedright
Indep
\end{minipage}
&
\begin{minipage}[b]{\linewidth}\raggedright
R2
\end{minipage}
&
\begin{minipage}[b]{\linewidth}\raggedright
Effect
\end{minipage}
&
\begin{minipage}[b]{\linewidth}\raggedright
Site
\end{minipage} \\
\midrule\noalign{}
\endhead
\bottomrule\noalign{}
\endlastfoot
h & netrad
& 0.51 &
\begin{minipage}[t]{\linewidth}\raggedright
\begin{verbatim}
  69
\end{verbatim}
\end{minipage}
&
BE-Lon \\
le &
netrad &
0.5 &
\begin{minipage}[t]{\linewidth}\raggedright
\begin{verbatim}
  38
\end{verbatim}
\end{minipage}
&
BE-Lon \\
le & rh &
0.31 &
\begin{minipage}[t]{\linewidth}\raggedright
\begin{verbatim}
  47
\end{verbatim}
\end{minipage}
&
BE-Lon \\
le & ta &
0.29 &
\begin{minipage}[t]{\linewidth}\raggedright
\begin{verbatim}
  51
\end{verbatim}
\end{minipage}
&
BE-Lon \\
fc &
netrad &
0.24 &
\begin{minipage}[t]{\linewidth}\raggedright
\begin{verbatim}
  22
\end{verbatim}
\end{minipage}
&
BE-Lon \\
co2 & ta &
0.22 &
\begin{minipage}[t]{\linewidth}\raggedright
\begin{verbatim}
  82
\end{verbatim}
\end{minipage}
&
BE-Lon \\
h & rh &
0.16 &
\begin{minipage}[t]{\linewidth}\raggedright
\begin{verbatim}
  59
\end{verbatim}
\end{minipage}
&
BE-Lon \\
h & ta &
0.15 &
\begin{minipage}[t]{\linewidth}\raggedright
\begin{verbatim}
  25
\end{verbatim}
\end{minipage}
&
BE-Lon \\
co2 & rh &
0.12 &
\begin{minipage}[t]{\linewidth}\raggedright
\begin{verbatim}
  14
\end{verbatim}
\end{minipage}
&
BE-Lon \\
co2 &
netrad &
0.12 &
\begin{minipage}[t]{\linewidth}\raggedright
\begin{verbatim}
  54
\end{verbatim}
\end{minipage}
&
BE-Lon \\
fc & ta &
0.11 &
\begin{minipage}[t]{\linewidth}\raggedright
\begin{verbatim}
  31
\end{verbatim}
\end{minipage}
&
BE-Lon \\
fc & rh &
0.07 &
\begin{minipage}[t]{\linewidth}\raggedright
\begin{verbatim}
   8
\end{verbatim}
\end{minipage}
&
BE-Lon \\
h & ws &
0.02 &
\begin{minipage}[t]{\linewidth}\raggedright
\begin{verbatim}
  60
\end{verbatim}
\end{minipage}
&
BE-Lon \\
fc & ws &
0.02 &
\begin{minipage}[t]{\linewidth}\raggedright
\begin{verbatim}
  94
\end{verbatim}
\end{minipage}
&
BE-Lon \\
le & p &
0.01 &
\begin{minipage}[t]{\linewidth}\raggedright
\begin{verbatim}
  49
\end{verbatim}
\end{minipage}
&
BE-Lon \\
le & ws &
0.01 &
\begin{minipage}[t]{\linewidth}\raggedright
\begin{verbatim}
  93
\end{verbatim}
\end{minipage}
&
BE-Lon \\
le & pa &
0.01 &
\begin{minipage}[t]{\linewidth}\raggedright
\begin{verbatim}
  98
\end{verbatim}
\end{minipage}
&
BE-Lon \\
h & netrad
& 0.67 &
\begin{minipage}[t]{\linewidth}\raggedright
\begin{verbatim}
   1
\end{verbatim}
\end{minipage}
&
BE-Vie \\
le &
netrad &
0.44 &
\begin{minipage}[t]{\linewidth}\raggedright
\begin{verbatim}
  25
\end{verbatim}
\end{minipage}
&
BE-Vie \\
fc &
netrad &
0.37 &
\begin{minipage}[t]{\linewidth}\raggedright
\begin{verbatim}
  35
\end{verbatim}
\end{minipage}
&
BE-Vie \\
h & rh &
0.36 &
\begin{minipage}[t]{\linewidth}\raggedright
\begin{verbatim}
  40
\end{verbatim}
\end{minipage}
&
BE-Vie \\
le & ta &
0.31 &
\begin{minipage}[t]{\linewidth}\raggedright
\begin{verbatim}
   9
\end{verbatim}
\end{minipage}
&
BE-Vie \\
le & rh &
0.21 &
\begin{minipage}[t]{\linewidth}\raggedright
\begin{verbatim}
  46
\end{verbatim}
\end{minipage}
&
BE-Vie \\
fc & ta &
0.21 &
\begin{minipage}[t]{\linewidth}\raggedright
\begin{verbatim}
  38
\end{verbatim}
\end{minipage}
&
BE-Vie \\
h & ta &
0.17 &
\begin{minipage}[t]{\linewidth}\raggedright
\begin{verbatim}
  57
\end{verbatim}
\end{minipage}
&
BE-Vie \\
fc & rh &
0.16 &
\begin{minipage}[t]{\linewidth}\raggedright
\begin{verbatim}
  33
\end{verbatim}
\end{minipage}
&
BE-Vie \\
co2 & ta &
0.13 &
\begin{minipage}[t]{\linewidth}\raggedright
\begin{verbatim}
  45
\end{verbatim}
\end{minipage}
&
BE-Vie \\
co2 & rh &
0.1 &
\begin{minipage}[t]{\linewidth}\raggedright
\begin{verbatim}
  15
\end{verbatim}
\end{minipage}
&
BE-Vie \\
co2 &
netrad &
0.07 &
\begin{minipage}[t]{\linewidth}\raggedright
\begin{verbatim}
  70
\end{verbatim}
\end{minipage}
&
BE-Vie \\
co2 & pa &
0.02 &
\begin{minipage}[t]{\linewidth}\raggedright
\begin{verbatim}
  41
\end{verbatim}
\end{minipage}
&
BE-Vie \\
co2 & ws &
0.01 &
\begin{minipage}[t]{\linewidth}\raggedright
\begin{verbatim}
  45
\end{verbatim}
\end{minipage}
&
BE-Vie \\
fc & p &
0.01 &
\begin{minipage}[t]{\linewidth}\raggedright
\begin{verbatim}
  39
\end{verbatim}
\end{minipage}
&
BE-Vie \\
le & p &
0.01 &
\begin{minipage}[t]{\linewidth}\raggedright
\begin{verbatim}
  74
\end{verbatim}
\end{minipage}
&
BE-Vie \\
h & ws &
0.01 &
\begin{minipage}[t]{\linewidth}\raggedright
\begin{verbatim}
  83
\end{verbatim}
\end{minipage}
&
BE-Vie \\
le & pa &
0.01 &
\begin{minipage}[t]{\linewidth}\raggedright
\begin{verbatim}
  95
\end{verbatim}
\end{minipage}
&
BE-Vie \\
le & ws &
0.01 &
\begin{minipage}[t]{\linewidth}\raggedright
\begin{verbatim}
  98
\end{verbatim}
\end{minipage}
&
BE-Vie \\
h & netrad
& 0.46 &
\begin{minipage}[t]{\linewidth}\raggedright
\begin{verbatim}
  98
\end{verbatim}
\end{minipage}
&
BE-Bra \\
h & rh &
0.26 &
\begin{minipage}[t]{\linewidth}\raggedright
\begin{verbatim}
  47
\end{verbatim}
\end{minipage}
&
BE-Bra \\
le &
netrad &
0.18 &
\begin{minipage}[t]{\linewidth}\raggedright
\begin{verbatim}
  52
\end{verbatim}
\end{minipage}
&
BE-Bra \\
le & ta &
0.14 &
\begin{minipage}[t]{\linewidth}\raggedright
\begin{verbatim}
  34
\end{verbatim}
\end{minipage}
&
BE-Bra \\
h & ta &
0.13 &
\begin{minipage}[t]{\linewidth}\raggedright
\begin{verbatim}
  21
\end{verbatim}
\end{minipage}
&
BE-Bra \\
le & rh &
0.12 &
\begin{minipage}[t]{\linewidth}\raggedright
\begin{verbatim}
  14
\end{verbatim}
\end{minipage}
&
BE-Bra \\
fc &
netrad &
0.07 &
\begin{minipage}[t]{\linewidth}\raggedright
\begin{verbatim}
  37
\end{verbatim}
\end{minipage}
&
BE-Bra \\
co2 & ta &
0.06 &
\begin{minipage}[t]{\linewidth}\raggedright
\begin{verbatim}
  67
\end{verbatim}
\end{minipage}
&
BE-Bra \\
fc & rh &
0.04 &
\begin{minipage}[t]{\linewidth}\raggedright
\begin{verbatim}
   7
\end{verbatim}
\end{minipage}
&
BE-Bra \\
co2 & rh &
0.02 &
\begin{minipage}[t]{\linewidth}\raggedright
\begin{verbatim}
  29
\end{verbatim}
\end{minipage}
&
BE-Bra \\
fc & ta &
0.02 &
\begin{minipage}[t]{\linewidth}\raggedright
\begin{verbatim}
  42
\end{verbatim}
\end{minipage}
&
BE-Bra \\
co2 &
netrad &
0.01 &
\begin{minipage}[t]{\linewidth}\raggedright
\begin{verbatim}
   7
\end{verbatim}
\end{minipage}
&
BE-Bra \\
co2 & ws &
0.01 &
\begin{minipage}[t]{\linewidth}\raggedright
\begin{verbatim}
  58
\end{verbatim}
\end{minipage}
&
BE-Bra \\
h & p &
0.01 &
\begin{minipage}[t]{\linewidth}\raggedright
\begin{verbatim}
  67
\end{verbatim}
\end{minipage}
&
BE-Bra \\
fc & pa &
0.01 &
\begin{minipage}[t]{\linewidth}\raggedright
\begin{verbatim}
  96
\end{verbatim}
\end{minipage}
&
BE-Bra \\
\end{longtable}


\noindent Table S2: All possible pairwise combinations of dependent and independent variables for the Fukushima case study (n=51). $R^2$ is the coefficient of determination of a linear model fit to the two covariates for the period of the abnormal release event (corresponding to 'during' in Fig. 2). Effect is the event effect size of the interaction term 'during' the known release period matching the quantile of the distribution as specified by event quantile effect hyperparameter.

\pagenumbering{gobble}

\begin{longtable}[]{@{}
  >{\raggedright\arraybackslash}p{(\linewidth - 8\tabcolsep) * \real{0.1667}}
  >{\raggedright\arraybackslash}p{(\linewidth - 8\tabcolsep) * \real{0.2083}}
  >{\raggedright\arraybackslash}p{(\linewidth - 8\tabcolsep) * \real{0.1458}}
  >{\raggedright\arraybackslash}p{(\linewidth - 8\tabcolsep) * \real{0.2292}}
  >{\raggedright\arraybackslash}p{(\linewidth - 8\tabcolsep) * \real{0.1875}}@{}}
\toprule\noalign{}
\begin{minipage}[b]{\linewidth}\raggedright
Dep
\end{minipage}
&
\begin{minipage}[b]{\linewidth}\raggedright
Indep
\end{minipage}
&
\begin{minipage}[b]{\linewidth}\raggedright
R2
\end{minipage}
&
\begin{minipage}[b]{\linewidth}\raggedright
Effect
\end{minipage}
&
\begin{minipage}[b]{\linewidth}\raggedright
Site
\end{minipage} \\
\midrule\noalign{}
\endhead
\bottomrule\noalign{}
\endlastfoot
h & netrad
& 0.73 &
\begin{minipage}[t]{\linewidth}\raggedright
\begin{verbatim}
  95
\end{verbatim}
\end{minipage}
&
OZ-Mul \\
h & rh &
0.2 &
\begin{minipage}[t]{\linewidth}\raggedright
\begin{verbatim}
  29
\end{verbatim}
\end{minipage}
&
OZ-Mul \\
le & rh &
0.15 &
\begin{minipage}[t]{\linewidth}\raggedright
\begin{verbatim}
  58
\end{verbatim}
\end{minipage}
&
OZ-Mul \\
h & ws &
0.12 &
\begin{minipage}[t]{\linewidth}\raggedright
\begin{verbatim}
  92
\end{verbatim}
\end{minipage}
&
OZ-Mul \\
h & ta &
0.08 &
\begin{minipage}[t]{\linewidth}\raggedright
\begin{verbatim}
   6
\end{verbatim}
\end{minipage}
&
OZ-Mul \\
le &
netrad &
0.07 &
\begin{minipage}[t]{\linewidth}\raggedright
\begin{verbatim}
  76
\end{verbatim}
\end{minipage}
&
OZ-Mul \\
co2 & ta &
0.05 &
\begin{minipage}[t]{\linewidth}\raggedright
\begin{verbatim}
  76
\end{verbatim}
\end{minipage}
&
OZ-Mul \\
co2 & rh &
0.03 &
\begin{minipage}[t]{\linewidth}\raggedright
\begin{verbatim}
  71
\end{verbatim}
\end{minipage}
&
OZ-Mul \\
le & ws &
0.01 &
\begin{minipage}[t]{\linewidth}\raggedright
\begin{verbatim}
  48
\end{verbatim}
\end{minipage}
&
OZ-Mul \\
le & p &
0.01 &
\begin{minipage}[t]{\linewidth}\raggedright
\begin{verbatim}
  63
\end{verbatim}
\end{minipage}
&
OZ-Mul \\
h & p &
0.01 &
\begin{minipage}[t]{\linewidth}\raggedright
\begin{verbatim}
  64
\end{verbatim}
\end{minipage}
&
OZ-Mul \\
h & netrad
& 0.76 &
\begin{minipage}[t]{\linewidth}\raggedright
\begin{verbatim}
   1
\end{verbatim}
\end{minipage}
&
US-Gle \\
le &
netrad &
0.33 &
\begin{minipage}[t]{\linewidth}\raggedright
\begin{verbatim}
  85
\end{verbatim}
\end{minipage}
&
US-Gle \\
fc & ta &
0.28 &
\begin{minipage}[t]{\linewidth}\raggedright
\begin{verbatim}
  49
\end{verbatim}
\end{minipage}
&
US-Gle \\
fc &
netrad &
0.25 &
\begin{minipage}[t]{\linewidth}\raggedright
\begin{verbatim}
   8
\end{verbatim}
\end{minipage}
&
US-Gle \\
fc & pa &
0.2 &
\begin{minipage}[t]{\linewidth}\raggedright
\begin{verbatim}
  47
\end{verbatim}
\end{minipage}
&
US-Gle \\
le & ta &
0.13 &
\begin{minipage}[t]{\linewidth}\raggedright
\begin{verbatim}
   6
\end{verbatim}
\end{minipage}
&
US-Gle \\
h & rh &
0.13 &
\begin{minipage}[t]{\linewidth}\raggedright
\begin{verbatim}
  87
\end{verbatim}
\end{minipage}
&
US-Gle \\
le & pa &
0.1 &
\begin{minipage}[t]{\linewidth}\raggedright
\begin{verbatim}
  28
\end{verbatim}
\end{minipage}
&
US-Gle \\
h & ta &
0.1 &
\begin{minipage}[t]{\linewidth}\raggedright
\begin{verbatim}
   2
\end{verbatim}
\end{minipage}
&
US-Gle \\
fc & ws &
0.07 &
\begin{minipage}[t]{\linewidth}\raggedright
\begin{verbatim}
  35
\end{verbatim}
\end{minipage}
&
US-Gle \\
fc & rh &
0.07 &
\begin{minipage}[t]{\linewidth}\raggedright
\begin{verbatim}
  26
\end{verbatim}
\end{minipage}
&
US-Gle \\
co2 & pa &
0.05 &
\begin{minipage}[t]{\linewidth}\raggedright
\begin{verbatim}
  59
\end{verbatim}
\end{minipage}
&
US-Gle \\
h & pa &
0.05 &
\begin{minipage}[t]{\linewidth}\raggedright
\begin{verbatim}
  84
\end{verbatim}
\end{minipage}
&
US-Gle \\
le & rh &
0.04 &
\begin{minipage}[t]{\linewidth}\raggedright
\begin{verbatim}
  71
\end{verbatim}
\end{minipage}
&
US-Gle \\
h & p &
0.04 &
\begin{minipage}[t]{\linewidth}\raggedright
\begin{verbatim}
  76
\end{verbatim}
\end{minipage}
&
US-Gle \\
co2 & ta &
0.04 &
\begin{minipage}[t]{\linewidth}\raggedright
\begin{verbatim}
  15
\end{verbatim}
\end{minipage}
&
US-Gle \\
co2 &
netrad &
0.02 &
\begin{minipage}[t]{\linewidth}\raggedright
\begin{verbatim}
   9
\end{verbatim}
\end{minipage}
&
US-Gle \\
fc & p &
0.02 &
\begin{minipage}[t]{\linewidth}\raggedright
\begin{verbatim}
  80
\end{verbatim}
\end{minipage}
&
US-Gle \\
co2 & ws &
0.02 &
\begin{minipage}[t]{\linewidth}\raggedright
\begin{verbatim}
  56
\end{verbatim}
\end{minipage}
&
US-Gle \\
co2 & rh &
0.01 &
\begin{minipage}[t]{\linewidth}\raggedright
\begin{verbatim}
  71
\end{verbatim}
\end{minipage}
&
US-Gle \\
h & rh &
0.33 &
\begin{minipage}[t]{\linewidth}\raggedright
\begin{verbatim}
  56
\end{verbatim}
\end{minipage}
&
US-Wrc \\
h & ta &
0.31 &
\begin{minipage}[t]{\linewidth}\raggedright
\begin{verbatim}
  51
\end{verbatim}
\end{minipage}
&
US-Wrc \\
h & ws &
0.16 &
\begin{minipage}[t]{\linewidth}\raggedright
\begin{verbatim}
  96
\end{verbatim}
\end{minipage}
&
US-Wrc \\
le & ta &
0.15 &
\begin{minipage}[t]{\linewidth}\raggedright
\begin{verbatim}
  27
\end{verbatim}
\end{minipage}
&
US-Wrc \\
le & rh &
0.14 &
\begin{minipage}[t]{\linewidth}\raggedright
\begin{verbatim}
  97
\end{verbatim}
\end{minipage}
&
US-Wrc \\
le & ws &
0.1 &
\begin{minipage}[t]{\linewidth}\raggedright
\begin{verbatim}
  80
\end{verbatim}
\end{minipage}
&
US-Wrc \\
co2 & ws &
0.04 &
\begin{minipage}[t]{\linewidth}\raggedright
\begin{verbatim}
  24
\end{verbatim}
\end{minipage}
&
US-Wrc \\
co2 & pa &
0.04 &
\begin{minipage}[t]{\linewidth}\raggedright
\begin{verbatim}
  28
\end{verbatim}
\end{minipage}
&
US-Wrc \\
h & p &
0.04 &
\begin{minipage}[t]{\linewidth}\raggedright
\begin{verbatim}
  75
\end{verbatim}
\end{minipage}
&
US-Wrc \\
fc & ws &
0.04 &
\begin{minipage}[t]{\linewidth}\raggedright
\begin{verbatim}
  82
\end{verbatim}
\end{minipage}
&
US-Wrc \\
co2 & ta &
0.03 &
\begin{minipage}[t]{\linewidth}\raggedright
\begin{verbatim}
  12
\end{verbatim}
\end{minipage}
&
US-Wrc \\
le & p &
0.02 &
\begin{minipage}[t]{\linewidth}\raggedright
\begin{verbatim}
  68
\end{verbatim}
\end{minipage}
&
US-Wrc \\
fc & rh &
0.02 &
\begin{minipage}[t]{\linewidth}\raggedright
\begin{verbatim}
  96
\end{verbatim}
\end{minipage}
&
US-Wrc \\
co2 & rh &
0.01 &
\begin{minipage}[t]{\linewidth}\raggedright
\begin{verbatim}
  62
\end{verbatim}
\end{minipage}
&
US-Wrc \\
le & pa &
0.01 &
\begin{minipage}[t]{\linewidth}\raggedright
\begin{verbatim}
  83
\end{verbatim}
\end{minipage}
&
US-Wrc \\
fc & p &
0.01 &
\begin{minipage}[t]{\linewidth}\raggedright
\begin{verbatim}
  83
\end{verbatim}
\end{minipage}
&
US-Wrc \\
\end{longtable}


\end{document}