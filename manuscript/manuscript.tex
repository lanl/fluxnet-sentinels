\documentclass{article}

\usepackage{arxiv}

\usepackage[utf8]{inputenc} % allow utf-8 input
\usepackage[T1]{fontenc}    % use 8-bit T1 fonts
\usepackage{hyperref}       % hyperlinks
\usepackage{url}            % simple URL typesetting
\usepackage{booktabs}       % professional-quality tables
\usepackage{amsfonts}       % blackboard math symbols
\usepackage{microtype}      % microtypography
\usepackage{graphicx}
\usepackage{natbib}
\usepackage{doi}
\usepackage{listings}
\usepackage{lineno}
\usepackage{setspace}
\usepackage{gensymb}

\linenumbers

\title{Eddy covariance towers as sentinels of abnormal radioactive material releases}

%\date{September 9, 1985}	% Here you can change the date presented in the paper title
\date{} 					% Or removing it

\author{
	\href{https://orcid.org/0000-0002-5924-2464}{\includegraphics[scale=0.06]{orcid.pdf}\hspace{1mm}Corresponding Author: Jemma Stachelek$^1$} \\
	$^1$ High Performance Computing Division\\
	Los Alamos National Laboratory\\
	Los Alamos, NM, USA, 87544 \\
	\texttt{jsta@lanl.gov} \\
	\And
	Vachel A. Kraklow$^2$ \\
	$^2$ Earth and Environmental Sciences Division\\
	Los Alamos National Laboratory\\
	Los Alamos, NM, USA, 87544 \\
	\And
	Elizabeth Christi Thompson$^3$ \\
	$^3$ Computer, Computational, and Statistical Sciences Division\\
	Los Alamos National Laboratory\\
	Los Alamos, NM, USA, 87544 \\
	\And
	\textbf{Lee Turin Dickman}$^2$ \\
	$^2$ Earth and Environmental Sciences Division\\
	Los Alamos National Laboratory\\
	Los Alamos, NM, USA, 87544 \\
	\And
	\textbf{Emily Casleton}$^3$ \\
	$^3$ Computer, Computational, and Statistical Sciences Division\\
	Los Alamos National Laboratory\\
	Los Alamos, NM, USA, 87544 \\
	\And
	\textbf{Sanna Sevanto}$^2$ \\
	$^2$ Earth and Environmental Sciences Division\\
	Los Alamos National Laboratory\\
	Los Alamos, NM, USA, 87544 \\
	\And
	\textbf{Ann Junghans}$^4$ \\
	$^4$ Nuclear and Engineering Nonproliferation Division\\
	Los Alamos National Laboratory\\
	Los Alamos, NM, USA, 87544
}

% Uncomment to remove the date
%\date{}

% Uncomment to override  the `A preprint' in the header
\renewcommand{\headeright}{\href{https://doi.org/}{DOI: 10.000/XXXXX}}
\renewcommand{\undertitle}{This is a draft of a Manuscript intended for Environmental Science and Pollution Research \\ \href{https://doi.org/10.1000/XXXXX}{DOI: 10.1000/XXXXX}}
\renewcommand{\shorttitle}{Eddy covariance towers as sentinels}

%%% Add PDF metadata to help others organize their library
%%% Once the PDF is generated, you can check the metadata with
%%% $ pdfinfo template.pdf
\hypersetup{
pdftitle={Eddy covariance towers as sentinels of abnormal radioactive material releases},
pdfsubject={stat.AP},
pdfauthor={Jemma Stachelek}
}

\begin{document}
\maketitle
\doublespacing

\vspace{-5em}

\section*{Abstract}	
Ensuring accurate detection and attribution of abnormal releases of radioactive material is critical for protecting human health and safety. Most commonly, such detection is accomplished via active monitoring approaches involving the collection of physical samples. This is labor intensive and limits the temporal and spatial resolution of any detected events to a relatively coarse level. As an alternative first step towards passive monitoring, we developed an approach using eddy flux tower data records to identify signals from a known abnormal release and quantify the extent to which that signal also occurs at other times in the data record. Through two case-studies, one of which targeted the Fukushima nuclear disaster and the other targeting an abnormal release event at a radioisotope production facility in Fleurus Belgium, we tested our approach and identified several potential heretofore unidentified abnormal events that were consistent with atmospheric circulation patterns and/or wind direction from known release sites. Because our approach is relatively simple and is resistant to systematic errors in the observational record, it has broad applicability beyond specific constituents and ecosystem types to identify a wide variety of limited-duration anomalies in flux tower data to ensure human health and industrial safety.

\textbf{Keywords:} eddy covariance, data mining, industrial safety, vegetation response, radioactivity, photosynthesis

\section{Introduction}
Detecting abnormal releases of radioactive material (hereafter simply “abnormal releases”) is important as a means of ensuring human health and industrial safety. The detection of such releases is typically accomplished using active approaches that require site-specific sampling or trace gas measurements \citep{loyalkaMechanicsAerosolsNuclear1983} sometimes as a part of international monitoring networks \citep[e.g.,][]{sangiorgiEuropeanRadiologicalData2020}. A potential shortcoming of these active monitoring approaches is that they require collection of physical samples (particulates, gasses, and other signal carriers) to retrospectively determine the nature of the release and the events that may have led up to it. Active monitoring with such means is thus labor intensive and limits the temporal and spatial resolution of any detected events to a relatively coarse level. 

A potential alternative to active monitoring is passive monitoring \citep[e.g., the German Integrated Measuring and Information system and the International Monitoring System,][]{bieringerMonitoringGroundlevelAir2004}. Although existing wide area monitoring systems are already critical components of abnormal release detection and tracking programs \citep{mediciIMSRadionuclideNetwork2001}, diversification-of-approach may increase the programs' sensitivity and comprehensiveness. With a greater diversity of methods, monitoring programs may be able to detect a wider array of constituents with differing characteristics or expand their geographic extent without the need to deploy new instrumentation. In the present study, we developed a novel passive monitoring approach to supplement existing programs that involves the use of eddy covariance data to capture a signal from known abnormal releases. This signal is then used to investigate whether additional potential abnormal releases exist at other times in the data record. We focused on eddy flux towers, as they have the potential to provide abnormal release signatures, are globally distributed, have been in near continuous operation for several decades, and can be relatively cheaply and easily deployed in locations of interest. Therefore, they constitute a rich target for data mining approaches aimed at identifying signatures resulting from direct interference of radioactive material with target measurements and/or the indirect signal of vegetation responses to the deposition of radioactive material.

Our utilization of flux tower records for event discrimination stands in contrast to the typical use of data from the towers for direct carbon balance accounting. Whereas carbon balance accounting activities deal directly in the units of the measured data for computing quantities like net ecosystem exchange \citep{baldocchiInterannualVariabilityNet2018}, our use treats each variable as potentially including an indirect signal (i.e. an anomaly from normal vegetation behavior) of radioactive material deposition irrespective of its physical properties (e.g., mass, volume, etc). In this way, we are not directly measuring radioactive deposition but rather leveraging the fact that existing ecosystem monitoring records reflect vegetation photosynthesis rates and general stress condition of site vegetation in a known way \citep{nobelPhysicochemicalEnvironmentalPlant2020}.

We specifically focused on short-term anomalies in the vegetation response data (on the order of days to weeks) due to either potential deposition of radioactive material or direct sensor effects rather than chronic long-term signals on the order of years to decades. In the first case study, we tested the ability of various data mining techniques to recover the signal from the closest tower (approximately 18 km away) to a known release coming from the Institut des Radioelements (IRE) in Belgium (hereafter called the known release location or the “facility”) nominally occurring on August 23, 2008. The IRE release originated from waste tanks into the atmosphere and lasted for several days totaling 50 GBq $^{131}I$ \citep{carle2010individual}. For example, grass samples collected in the vicinity of the facility found radioactivity levels of 5000 Bq/kg. In the second case study, we further explored potential signals in flux tower data related to the Fukushima Nuclear Disaster nominally occurring on March 11, 2011 releasing a massive radioactive plume that reached North America in 5 days that eventually dispersed throughout the entire northern hemisphere \citep{meszarosPredictabilityDispersionFukushimaderived2016}. We used the signal identified during the known release to identify additional potential abnormal releases occurring at other times in the data record and at other eddy flux towers. While the body of literature on direct or indirect effects of radioactive material on vegetation is limited, prior studies support the idea that exposure may cause declines in photosynthetic $CO_2$ uptake. For example, \citet{gudkovEffectIonizingRadiation2019} detail how a wide range of exposure doses and durations can inhibit CO2 uptake as a result of enzymatic effects on the Calvin Cycle. For the present study, we lack leaf-level observations to directly establish inhibition and instead express it as a change in the slope of the $CO_2$ relationship with other environmental variables (e.g. air temperature, humidity, precipitation, etc.) before, during, and after exposure.


\section{Methods}

\subsection{Data Description}

For both the IRE and Fukushima case studiesy, we gathered measurements from eddy covariance platforms (i.e., “flux towers”) spanning multiple years. For the IRE case study, we targeted towers located in Belgium, which are part of the ICOS (Integrated Carbon Observation System) network available through the European Fluxes Database (http://www.europe-fluxdata.eu/). Three sites in proximity to the known release location had sufficient long-term data for our anticipated data mining efforts (BE-Lon, BE-Bra, and BE-Vie sites, Figure 1A). Each of the three flux tower sites are located within a crop ecosystem type and had almost 10 years of data available from 2004 through 2013 recorded at 30 min intervals. While BE-Lon had a near-continuous data record during this period, BE-Bra had a long period of missing data in 2005, and both BE-Bra and BE-Vie had a long period of missing data in 2009. The distance between IRE and the flux tower sites are approximately 19 km, 95 km, and 105 km for the BE-Lon, BE-Bra, and BE-Vie, respectively.

For the Fukushima case study, due to the direction of the plume, we gathered measurements from flux towers located in the Northern Hemisphere (Western United States), which are part of the Ameriflux network (https://ameriflux.lbl.gov/). To increase confidence that we are seeing a real effect, we contrasted the results from Northern Hemisphere sites, where we expect a higher likelihood of impact, to measurements from flux towers located in the Southern Hemisphere, which are part of the OzFlux network \citep{cleverly2011alice}. We selected sites with sufficient long-term data to cover the 2011 date of the disaster and excluded boreal sites with extreme snow cover related seasonality. We ultimately selected three sites (US-Wrc, US-GLE, and OZ-Mul, Figure 1B).

\subsection{Statistical Approach}
We processed each data file by 1) excluding nighttime measurements based on photosynthetic photon flux density (PPFD > 100 µmol m-2 s-1) or net solar radiation (netrad > 0 $W/m^2$) and 2) defining an “event period” following the publicly released date of the abnormal release. We focused on the following variables as potentially containing a signature of the release : carbon dioxide, carbon dioxide flux (fc), latent heat (le), and sensible heat flux (h). Conversely, the following were treated as independent explanatory variables: wind speed, precipitation, atmospheric pressure, relative humidity, photosynthetic photon flux density, air temperature, and net radiation. Additionally, we used wind direction data from the eddy flux towers to determine when or if any released material might plausibly reach a particular location. All of the aforementioned data for each of these variables obtained from flux tower instrumentation recorded at a 30 minute (min) interval barring any gaps due to missing data.

To begin our analysis, we screened all pairwise, linear relationships between the aforementioned independent and dependent variables using a linear regression fit between each pair for the entire period of record (n=51). Variable pairs were excluded from further analysis if the overall coefficient of determination ($R^2$) was less than 0.1. This low threshold was chosen to maximize the number of pairs that could potentially be investigated more deeply. From visual inspection, we determined that many of the rejected variable pairs appeared to have a nonlinear relationship. As our aim was to develop a methodology that could be applied broadly to other sites, we did not apply transformation to linearize these relationships without a physical justification for the transformation. Evaluating an exhaustive list of data transformations was beyond the scope of the present effort. After identification of candidate variable pairs, we fit an interaction model of the following form where Y represents a continuous response (e.g., $CO_2$, latent heat, etc.), $X_1$ is an independent variable (e.g., air temperature, humidity, etc.), $W_2$ is a categorical variable representing the “period” of observation (i.e., during, before, or after, the event), $\beta$0 is the intercept, $\beta$1,2 are slope parameters for $X_1$ and $W_2$ respectively, and $\beta$3 is the slope parameter for the interaction between $X_1$ and $W_2$:

\[ Y = \beta_0 + \beta_1X_1 + \beta_2W_2 + \beta_3X_1W_2 \]

This interaction model was fit for a window of time encompassing the known release event and for every other possible window in the period-of-record. The overall length of the window was treated as an adjustable hyperparameter since the length of time for an abnormal material release to reach the footprint of the flux tower may be event-dependent and/or unknown. In addition, the length of time for deposited material to induce a possible vegetation response (or at least a flux data anomaly) is an unknown quantity. Thus, in the default base case, we set the window length (n\_days) as 7 days, but we ran hyperparameter experiments to determine the result of setting it to longer periods of 10 and 14 days. The window length affects the duration of time designated as before and after the event. The period of time designated as during the event was set as 2 days in all cases. We compared the effect size of the interaction term ($\beta$3) during the event-encompassing window against all other windows in a “rolling” analysis. Therefore, the overall analysis compares the interaction term of the window of the known event against a window centered on every other observation. Specific windows that had an effect size at least as large as that of the known event and with wind conditions defined as “towards” the flux tower were flagged as “event detections”.

In order to flag event detections, we defined several adjustable hyperparameters to account for uncertainty surrounding atmospheric transport and the timing of any possible responses to material deposition. The first hyperparameter we defined, wind\_tolerance, reflects the range around the bearing from the facility to the tower location which counts as towards the tower. In the base default case, we set wind\_tolerance to 10 degrees, but we explored setting it to smaller values down to 5 degrees and larger values up to 20 degrees. The second hyperparameter we defined, event\_quantile\_effect, handles the unknown travel time of any materials to be deposited via an abnormal release or what, if any, delays exist between deposition and vegetation response. Rather than simply selecting the maximum value of $\beta$3 during a given window or the value at the exact time of the event, the value of event\_quantile\_effect is set to correspond with a specific quantile of all $\beta$3 values during the event window. In the base default case, this was set at 0.9, but we explored values as low as 0.5. All statistical analyses were carried out using the statsmodels Python package (Seabold and Perktold, 2009). Our processing scripts are openly available at [doi link], and we refer readers to the original providers for data access.

\begin{figure}
	\centering
	\includegraphics[width=16cm]{__map}
	\caption{Map of flux tower locations for the IRE and Fukushima case studies (left and right respectively). Note that BE-Lon and BE-Vie are of a similar easterly direction to the IRE site whereas BE-Bra is in a northerly direction. The distance between IRE and the flux tower sites are approximately 19, 95, and 105 km for the BE-Lon, BE-Bra, and BE-Vie respectively. Each of the three IRE flux tower sites are located within a crop ecosystem type. Note that the US-Wrc site would be expected to have first contact with the Fukushima atmospheric plume followed by US-Gle. The OZ-Mul site would not be expected to have been affected by the plume.}
	\label{fig:study_site}
\end{figure}

\section{Results}

For the IRE case study, among all possible pairwise combinations of dependent and independent variables (n=51, Table S1), we found that only the $CO_2$ versus air temperature comparison had any measure of predictability from the model ($R^2$ > 0.05) and a strong event effect size (Table 1). Although other variable pairs such as sensible heat flux and net radiation had a stronger effect size for the BE-Lon site, they had a weak linear relationship across the other sites. When we focus on the period of the abnormal release event in particular rather than the overall period of record, we found that $CO_2$ versus air temperature relationship at each of the three focal towers had similar $R^2$ values and effect sizes (Table 1). Figure 2 provides a visual example of the effect size framing in our analysis. Note how the apparent interaction effect between time periods around the abnormal release event in Figure 2A (before, during, after) translates to a relatively high effect size (Figure 2C) whereas in Figure 2B at an arbitrary time point, there is no apparent interaction effect. This corresponds to a low effect size in Figure 2D. Note also how during and after the known abnormal release, the slopes of the relationships between $CO_2$ and air temperature become less negative than before the event, with lower than expected $CO_2$ at low air temperature (Figure 2A).

Despite the similar effect size at the three tower sites during the abnormal release event, they differed substantially in the degree to which they uniquely flagged the known release event. At the BE-Lon and BE-Vie sites, our event detection algorithm identified the known release and only a few other time periods as potential abnormal releases (Figure 3). Conversely, at the northerly BE-Bra site, the algorithm was not able to identify the known release event distinct from background noise. This can be seen in the rate of event detections ($\alpha$) at each site, whereby it was less than 1\% at the BE-Lon and BE-Vie sites, but was greater than 2\% at the BE-Bra site. It is notable that this ability to uniquely detect the known release along with a plausible number of additional event detections corresponds with the fraction of the abnormal release period when the wind was pointed from the release location towards the respective tower (using the default base case wind tolerance of 10 deg) which was much lower at the BE-Bra site (5\%) compared to the BE-Lon and BE-Vie sites (34\% Table 1). 

To increase our confidence that event detections were not dependent on the specific hyperparameters settings in our base-case setup, we ran an exhaustive cross-validation set of hyperparameter experiments. For this effort, we tested every possible combination of values for wind tolerance, number of days, and the event quantile effect. Across reasonable values of specific hyperparameters, we found that their magnitudes had little effect on the rate at which events were flagged as potential abnormal releases as even in the most extreme cases, we did not see detection rates exceeding 1\%. We did, however, observe that the event detection was related to the window length and event quantile parameters as shown by the slope of the line in Figure 4. Although increasing values of the event quantile parameter would hypothetically increase event detection rates, we do not see value in setting this parameter below the median. Similarly, increasing the magnitude of the window length parameter would also likely increase the event detection rate but here too we do not see value in setting this parameter beyond 14 days given the results of prior atmospheric modeling efforts \citep[e.g.][]{meszarosPredictabilityDispersionFukushimaderived2016}.

For the Fukushima case study, our objective was to test the sensitivity of our approach to a broad scale abnormal release and to introduce a control design to verify behavior on a flux tower where we expect no impact from any abnormal releases. First, we verified that event detection specificity is present at towers likely to be exposed even across long distances if the abnormal release is large (Figure 5). Then, we found a decrease in event detection specificity and an increase in event detection rate moving from the flux tower expected to be most affected (US-Wrc), to a more distant (likely less affected) tower (US-Gle), to a tower likely not exposed at all to the atmospheric plume (OZ-Mul, Figure 5).

\begin{table}[]
	\centering
	\caption{Overview of $CO_2$ vs air temperature interaction models at the IRE site. $R^2$ is the coefficient of determination of a linear model fit to the two covariates for the period of the abnormal release event (corresponding to “during” in Fig. 2). Effect is the event effect size of the interaction term 'during' the known release period matching the quantile of the distribution as specified by event quantile effect hyperparameter. Wind (\%) is the average percentage of time that wind direction was from the site of known release towards the selected flux tower during the known release event. The remaining columns indicate the base-case hyperparameter settings.}
	\begin{tabular}{@{}lllllll@{}}
	\toprule
	Site & $R^2$ & Effect & Wind(\%) & Days(n) & Wind(tol) & Effect(q) \\ \midrule
	BE-Vie & 0.35 & 36.2  & 34 & 7 & 10 & 0.9  \\
	BE-Bra & 0.4 & 57.59  & 5 & 7 & 10 & 0.9 \\
	BE-Lon & 0.47 & 30.13 & 34 & 7 & 10 & 0.9 \\ \bottomrule
	\end{tabular}	
	\label{table:1}
	\end{table}

\begin{figure}
	\centering
	\includegraphics[width=11cm]{__interaction_belon}
	\caption{Interaction plots at the BE-Lon site between air temperature ($C^{\circ}$) and $CO_2$ (ppm) for the period of the known abnormal release (A) and an arbitrary non-release period (B). Also shown is the distribution of all interaction effect sizes (gray) compared with the effect size of the time period shown in the corresponding left panels (orange, C, D).}
	\label{fig:interaction}
\end{figure}


\begin{figure}
	\centering
	\includegraphics[width=16cm]{__rolling_fleurus}
	\caption{Time series plots (2004-2013) of rolling event detection analysis at the IRE site for the $CO_2$ and air temperature ($C^{\circ}$) variable pair. Event detection lines (orange), interaction effect size (blue), and event effect size (dashed black line) are shown in the top panels. The fraction of time that the wind direction was towards the particular tower (solid black) are shown in the corresponding bottom panels.}
	\label{fig:rolling}
\end{figure}

\begin{figure}
	\centering
	\includegraphics[width=16cm]{__hyperparameter_experiment}
	\caption{Sensitivity of event detection at the IRE site to different hyperparameter values (points) alongside potential relationship (solid line) and confidence interval (shaded region).}
	\label{fig:hyperparameter}
\end{figure}

\begin{figure}
	\centering
	\includegraphics[width=16cm]{__rolling_fukushima}
	\caption{Time series plots (2010-2013) of rolling event detection analysis for the Fukushima disaster case study. The depicted variable pair is latent heat and relative humidity. Event detection lines (orange), interaction effect size (blue), and event effect size (dashed black line) are shown in the top panels. The fraction of time that the wind direction was towards the particular tower (solid black) are shown in the corresponding bottom panels.}
	\label{fig:fukushima}
\end{figure}

\section{Discussion}
Using flux tower records collected closest to and in the prevailing wind direction to the IRE and the Fukushima release locations, our event detection algorithm was able to identify both known abnormal releases and plausible previously unidentified abnormal events. There was broad agreement among the closest flux tower and more distant flux towers as to the timing and frequency of these releases, and our approach was supported by the loss of signal at flux towers outside of the prevailing wind direction. This suggests that, at the very least, given a facility known to have a prior abnormal release, flux tower networks are capable of providing a means of passive monitoring to detect subsequent events as well as information on the environmental conditions before, during, and after the event. Furthermore, our approach has promise for situations when there is not a prior known abnormal release. In these cases, it may be possible to continuously mine flux tower records (as they become available) for events that exceed some threshold identified in a more comprehensive cross-site, cross-event benchmarking study. The delay between data collection and availability to the general community is highly variable among sites and investigators. In the best case scenario, for actively monitored sites with responsive investigators, this can be as little as 6 months. As a result, operationalization of our approach might target post-hoc attribution rather than real-time monitoring applications.

More generally, we show that flux tower records have value for this type of data mining despite the apparent noisiness of the data \citep[e.g., ][]{fratiniEddyCovarianceFlux2018}. We attribute the sensitivity of our approach to the fact that we fit successive interaction models to limited sections of the data record, and we leveraged bivariate relationships to constrain noisiness in individual data records. Our approach was aided by the fact that this type of scale-free data mining is largely unaffected by systematic errors due to uncertainty in calibration standards and/or low resolution in specific sensor packages deployed on the towers. This is because although such systematic errors may bias the magnitude of overall flux values, which would be a problem for typical uses of flux towers that deal directly in the units of the measured data, they do not reduce the overall confidence in individual values \citep{langfordEddycovarianceDataLow2015}, which for our purposes might represent an abnormal release signal. The ability of our approach to identify a suitable release signature does likely depend on the quality and completeness of the underlying data.

Another contributor to the success of our approach was our ability to compare event detections at  close towers in prevailing wind directions against more distant towers (refer to Figure 1), increasing our confidence that we were seeing a true signal and not spurious noise. The relatively dense networks of flux towers with long data records leveraged in both locations also aided these case studies. This level of flux tower density may be found in the United States, Western/Central Europe, Japan, and Australia, which have high densities of flux towers, but not throughout the rest of the world \citep{baldocchi2001fluxnet,pastorello2020fluxnet2015}. As a result, this may impose geographical limits on our approach, although installation and maintenance of new flux towers in locations of interest may be cost effective compared to other monitoring approaches. Another attribute that likely aided our investigations was the fact that all three of the flux towers we examined were located in similar agricultural and forested ecosystem types for the IRE and Fukushima case studies respectively. Although investigating whether or not flux tower combinations with differing ecosystems would yield similar results is beyond the scope of the present study, we suspect that the data records among towers in disparate ecosystem types could differ too drastically to be of comparative use. For example, \citet{pastorello2020fluxnet2015} show that towers in the crop ecosystem type, of which all three sites used in this study belong, have a relatively narrow distribution of fluxes (i.e., gross primary production, GPP) compared to towers in the grassland or evergreen broadleaf forest types. One reason for this narrow spread in crop ecosystems may be the relative homogeneity of flux tower footprints in crop ecosystems, which typically extend to within 1000m of the tower depending on atmospheric conditions \citep[e.g., wind speed and direction,][]{chuRepresentativenessEddyCovarianceFlux2021}. Given that towers in other ecosystem types with a more heterogeneous flux tower footprint have a wider distribution of fluxes \citep[e.g.][]{pastorello2020fluxnet2015}, we suspect that having all the towers in the IRE case study located in a homogenous ecosystem type was helpful in reducing the spikiness and spread of the data. 

The biggest improvements to our approach would likely come from coupling our data mining procedure with rigorous air pollutant dispersion modeling \cite[e.g.,][]{meszarosPredictabilityDispersionFukushimaderived2016}. This would eliminate the need for indirect estimation of material transport via uncertainty analyses and instead use simulation results to analytically determine the likely arrival and duration of material deposition. Further improvements could be made by incorporating knowledge about the composition of materials being deposited \citep{international2006iaea, meszarosPredictabilityDispersionFukushimaderived2016} and/or the interactions between material exposure and other environmental stressors \citep{mousseauPlantsLightIonizing2020}, which likely affects the severity and timing of ecosystem responses and by extension the fluxes being measured by a given tower. Such information may help disentangle potential interference of atmospheric contaminants with infrared measurement of $CO_2$ from declines in photosynthetic $CO_2$ uptake of contaminated vegetation given our observation that during and after the known abnormal release, the slopes of the relationships between $CO_2$ and air temperature become less negative than before the event, with lower than expected $CO_2$ at low air temperature. Although the chemical makeup of possible atmospheric contaminants is unknown, interference by iodine itself is unlikely to affect measurements of $CO_2$ given that the absorption spectra of iodine peaks at much shorter wavelengths around 10-7 m \citep{haynes2016crc} compared to that of $CO_2$ which peaks around 1700-2100 10-9 m (LICOR Biosciences Inc., Lincoln, NE).

The fact that we were able to identify known releases (along with potential unidentified abnormal events) in the data records from both a large event (i.e. Fukushima) as well as a smaller event despite not having a detailed radiological or atmospheric transport model is a strength of our approach. Furthermore, our approach is not limited to a radiological context, any abnormal event that affects the plant community and is reflected in flux tower data is a potential target. Because our approach is light-weight and resistant to systematic errors in the observational record, it has broad applicability beyond specific constituents and ecosystem types to identify a wide variety of limited-duration impacts to the plant community within flux tower footprints to ensure human health and industrial safety.

% \section{Data Availability Statement}

% All code associated with this manuscript will be made available by the time of publication (or earlier pending the conclusion of US Department of Energy approvals) with a Zenodo DOI at: \href{https://doi.org}{https://doi.org}

\bibliographystyle{jsta}
\bibliography{fluxnet}

\section*{Statements and Declarations}

\subsection*{Funding}

We acknowledge funding for supporting the AmeriFlux data portal: U.S. Department of Energy Office of Science. The data used in this activity has been also funded by CarboEuropeIP (EU-FP6), IMECC (EU-FP6). This work was supported by Los Alamos National Laboratory (LDRD-20220062ER).

\subsection*{Competing Interests}

The authors have no relevant financial or non-financial interests to disclose.

\subsection*{Author Contributions}

JS built models, analyzed data, and wrote the paper. SS, LTD, and AJ contributed to the conception of the manuscript. ECT, VAK, LTD, and EC edited the manuscript. All authors provided interpretation of results.

\subsection*{Ethical Approval}

Not applicable

\subsection*{Consent to Participate}

Not applicable

\subsection*{Consent to Publish}

Not applicable

\subsection*{Availability of Data and Materials}

No new data was produced in this study. All original data used is available publicly from their respective sources and archived permanently with restrictions on redistribution. These can be found at \url{http://www.europe-fluxdata.eu/}, \url{https://fluxnet.org/}, and \citep{cleverly2011alice}.

Release of computer code and secondary data reuse that supports the results and analyses of the paper are pending a United State Department of Energy Review pursuant to \url{https://www.osti.gov/doecode/FAQs#how-do-i-announce-software-with-an-access-limitation-to-osti}.


\end{document}
